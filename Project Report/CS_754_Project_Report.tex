\documentclass[fleqn, 11pt]{article}

\usepackage{verbatim}
\usepackage{amsmath}
\DeclareMathOperator*{\argmin}{arg\,min}
\DeclareMathOperator*{\argmax}{arg\,max}
\usepackage{amssymb}
\usepackage{amsthm}
\usepackage{hyperref}
\usepackage{ulem}
\usepackage{enumitem}
\usepackage[left=0.75in, right=0.75in, bottom=0.75in]{geometry}
\usepackage{graphicx}

\newcommand{\myline}{
  \par
  \kern3pt % space above the rules
  \hrule height 0.5pt
  \kern2pt % space between the rules
  \hrule height 0.5pt
  \kern3pt % space below the rules
  \par
}

\usepackage[T1]{fontenc}

\newcommand{\bs}[1]{\boldsymbol{#1}}
\newcommand\norm[1]{\left\lVert#1\right\rVert}

\usepackage{array}
\usepackage{caption}
\usepackage{floatrow}
\usepackage{multirow}

\usepackage{chngcntr}
\counterwithin*{equation}{section}
\counterwithin*{equation}{subsection}

\usepackage{sectsty}
\sectionfont{\centering}

\usepackage[perpage]{footmisc}

\usepackage{fancyhdr}
\pagestyle{fancy}
\fancyhf{}
\lhead{190100036 \& 190100044}
\rhead{CS 754: Assignment 3}
\renewcommand{\footrulewidth}{1.0pt}
\cfoot{Page \thepage}

\setlength{\parindent}{0em}
\renewcommand{\arraystretch}{2}%

\title{CS 754: Advanced Image Processing \\ Project Report}
\author{ 
\begin{tabular}{|c|c|}
     \hline
     \textsf{Krushnakant Bhattad} & \textsf{ \hspace{5pt} Devansh Jain \hspace{5pt} } \\
     \hline
     \textsf{190100036} & \textsf{190100044}\\
     \hline
\end{tabular}
}
\date{May 11, 2021}

\begin{document}

\maketitle
\thispagestyle{empty}
\renewcommand{\arraystretch}{1}%

\myline 

\vspace{7pt}

\underline{\large {\textsc{Project Title}}}: 

\medskip  

Implementing Low-rank matrix completion algorithm for Video denoising and comparing it with other denoising algorithms like PCA and VBM3D method. 

\hrulefill 

\vspace{10pt}

\underline{\large {\textsc{Main Reference Paper}}}: 

\medskip 

"Robust video denoising using low rank matrix completion"

by Hui Ji, Chaoqiang Liu, Zuowei Shen, Yuhong Xu.

Link to the paper: \url{https://ieeexplore.ieee.org/document/5539849}

\hrulefill

\vspace{10pt}

\underline{\large {\textsc{Data set used}}}: 

\medskip 

The original data set can be accessed at: \url{https://media.xiph.org/video/derf/}

\hrulefill

\vspace{10pt}

\underline{\large {\textsc{Validation Strategy}}}: 

\medskip 

We compared our results to that from the methods with impulsive noise pre-processing, like PCA and VBM3D, with respect to their PSNR values (\underline{P}eak \underline{S}ignal to \underline{N}oise \underline{R}atio) and also visually.

\hrulefill

\vspace{10pt}

\underline{\large {\textsc{Associated GitHub Repository}}}:

\medskip

% The working code with results and report is present on the GitHub repository which can accessed at: \url{https://github.com/devansh-dvj/Video-denoising}
The GitHub repository can accessed at: \url{https://github.com/devansh-dvj/Video-denoising/}

\vspace{7pt}

\myline

\newpage
\vspace{-2em}
\myline

\vspace{10pt}

\underline{\large {\textsc{Abstract}}}: 

\medskip  

Most existing video denoising algorithms assume a single statistical model of image noise, e.g. additive Gaussian white noise, which often is violated in practice. The paper presented a new patch-based video denoising algorithm capable of removing serious mixed noise from the video data. \\

The principle is to group similar patches in both spatial and temporal domain to formulate the problem of removing mixed noise as a low-rank matrix completion problem, which leads to a denoising scheme without strong assumptions on the statistical properties of noise. The resulting nuclear norm related minimization problem can be efficiently solved using various techniques, one of which was implemented by us - Singular Value Thresholding algorithm.\\



\vspace{7pt}

\myline 
\end{document}

